% Version 2.21 of 2022/01/12
\documentclass[runningheads]{llncs}
%
\usepackage[T1]{fontenc}
\usepackage{graphicx}
\usepackage{epstopdf}
\usepackage{subcaption}
\usepackage{footnote}


% mathbb font
\usepackage{amssymb}
% C++ code
\usepackage{color}
\usepackage{listings}
\lstset{ %
language=C++,                % choose the language of the code
basicstyle=\small\ttfamily,       % the size of the fonts that are used for the code \footnotesize
numbers=left,                   % where to put the line-numbers
numberstyle=\small\ttfamily,      % the size of the fonts that are used for the line-numbers \footnotesize
stepnumber=1,                   % the step between two line-numbers. If it is 1 each line will be numbered
numbersep=5pt,                  % how far the line-numbers are from the code
backgroundcolor=\color{white},  % choose the background color. You must add \usepackage{color}
showspaces=false,               % show spaces adding particular underscores
showstringspaces=false,         % underline spaces within strings
showtabs=false,                 % show tabs within strings adding particular underscores
frame=single,           % adds a frame around the code
tabsize=2,          % sets default tabsize to 2 spaces
captionpos=b,           % sets the caption-position to bottom
breaklines=true,        % sets automatic line breaking
breakatwhitespace=false,    % sets if automatic breaks should only happen at whitespace
escapeinside={\%*}{*)},          % if you want to add a comment within your code
numberbychapter={false}         % default true
}
% fig
\captionsetup{compatibility=false}
%
\begin{document}
%
\title{EasyCompare: Performance Analysis System for GNN Models across GPUs}

%
%\titlerunning{Abbreviated paper title}
% If the paper title is too long for the running head, you can set
% an abbreviated paper title here

%\author{Kejie Ma\inst{1}$\dagger$ \and Zizheng Zhang\inst{1}$\dagger$ \and Qingxiao Sun\inst{1} \and Hailong Yang\inst{1}\thanks{Corresponding author, hailong.yang@buaa.edu.cn} \and Zhongzhi Luan\inst{1} \and Depei Qian\inst{1}}
%\authorrunning{K. Ma, Z. Zhang et al.}
%\institute{School of Computer Science and Engineering, Beihang University, Beijing, China
%\email{\{19373449,cpyking,qingxiaosun,hailong.yang,07680,depeiq\}@buaa.edu.cn}}

%\def\thefootnote{$\dagger$}\footnotetext{Equal contribution}\def\thefootnote{\arabic{footnote}}

%\author{Tianyu Feng$^{1,2}$ \and Siyan Chen$^{2}$ \and Xin You$^{2}$ \and Shuzhang Zhong$^{2}$ \and Hailong Yang$^{1,2}$\thanks{Corresponding author, hailong.yang@buaa.edu.cn} \and Zhongzhi Luan$^{2}$ \and Depei Qian$^{2}$}
%\authorrunning{T. Feng, S. Chen et al.}
%\institute{State Key Laboratory of Software Development Environment$^{1}$, Beijing, China \\
%School of Computer Science and Engineering, Beihang University$^{2}$, Beijing, China}

%\author{First Author\inst{1}\orcidID{0000-1111-2222-3333} \and
%Second Author\inst{2,3}\orcidID{1111-2222-3333-4444} \and
%Third Author\inst{3}\orcidID{2222--3333-4444-5555}}
%
%\authorrunning{F. Author et al.}
%
% First names are abbreviated in the running head.
% If there are more than two authors, 'et al.' is used.
%
%\institute{Princeton University, Princeton NJ 08544, USA \and
%Springer Heidelberg, Tiergartenstr. 17, 69121 Heidelberg, Germany
%\email{lncs@springer.com}\\
%\url{http://www.springer.com/gp/computer-science/lncs} \and
%ABC Institute, Rupert-Karls-University Heidelberg, Heidelberg, Germany\\
%\email{\{abc,lncs\}@uni-heidelberg.de}}
%
\maketitle              % typeset the header of the contribution
%
\begin{abstract}
  Over the years, the performance optimization of Graph Neural Netwoks (GNNs) has attracted attention. The factors affecting GNN performance include framework designs and hardware architectures (e.g. GPUs). Performance comparisons of GNN models between different GNN frameworks and GPUs are significant, while it's difficult to conduct fair comparison between GNN frameworks due to the inconsistent language styles between frameworks. To this end, we propose EasyComapre, an integrated system for efficient GNN performance comparison and result analysis. Specifically, EasyCompare first incorporates a Domain Specific Language (DSL) enabling GNN models definition. Second, EasyCompare integrates a code transformer to convert DSL into executable codes on different GNN frameworks of the same network structure. Third, EasyCompare automatically run GNN models and visualize logs for better analysis. We conduct extensive experiment to compare the performance of various GNN models between different frameworks and GPUs. The insights derived from the comparison result can help developers to explore optimization schemes for GNN frameworks.

  \keywords{Performance \and GNN frameworks \and Comparison.}
\end{abstract}
%
%
%
\section{Introduction}
\label{sec:introduction}
In recent years, Graph Neural Networks (GNNs) have been widely adopted in many aspects, such as node classification, edge classification, and link prediction. Different from traditional deep neural networks (DNNs) that operate on Euclidean data (e.g.,1D text and 2D images), GNNs are trained on non-Euclidean data (e.g., social-network graphs and knowledge graphs). Moreover, GNNs generally involve highly memory-intensive graph operations, making it challenging to achieve high performance on many-core processors (e.g., GPUs).

The performance optimization of GNNs has gradually attracted attention. The GNN performance is affected by various factors such as framework designs and GPU architectures. For example, the mainstream GNN frameworks, including PyTorch Geometric (PyG)~\cite{fey2019fast} and Deep Graph Library (DGL)~\cite{wang2019deep} have different graph operator implementations and layer optimization strategies. In addition, GPUs with varying architectures of hardware impact the performance of GNNs. For instance, memory bandwidth contention or cache conflicts may significantly degrade the GNN performance.

Multi-dimensional performance comparisons of GNN models are essential for both users and researchers. On the one hand, by comparing the performance of GNN models on different frameworks and GPUs, users can make the informed framework and GPU choices. On the other hand, the operator-level performance comparison of GNN frameworks helps researchers determine the framework implementation's potential impact. In this way, researchers can explore further optimization paths for GNN frameworks. However, the performance comparison of GNN models in frameworks and GPUs has the following challenges.

\textbf{Inconsistent language styles between frameworks.} PyG and DGL have different code organization and interface definitions. As a result, great manual efforts are required to make the same GNN structure functionally equivalent on different frameworks. The inconsistent language styles between frameworks prevent users from quickly achieving a fair comparison of framework performance.

\textbf{Lack of performance analysis systems.} A comprehensive performance comparison requires deploying GNN models implemented with multiple GNN frameworks on diverse GPUs. Furthermore, log recycling and result visualization are tedious and tricky for users and researchers. Finally, we implement an integrated system that can run GNN models on various GPUs and visualize performance logs for better comparison and analysis.

To this end, we propose EasyCompare to achieve high efficiency for GNN performance comparison and result analysis. First, we design a Domain Specific Language (DSL) called GUC (\underline{G}NN \underline{U}nified Language based on \underline{C}$++$), which enables the convenient definition of GNN models. Then, we develop a code transformer to automatically convert GUC DSL into executable Python codes on DGL and PyG frameworks. Finally, we implement an integrated system that runs GNN models on various GPUs and visualizes performance logs for better analysis. With EasyCompare, we conduct extensive experiments to deeply explore the factors that influence the performance of GNN models. The insights provide useful suggestions for optimizing GNN frameworks on GPUs.

%The rest of this paper is organized as follows. Section~\ref{sec:background} presents the background. Sections~\ref{sec:gnndsl}-\ref{sec:system} present the details of EasyCompare implementation. Section~\ref{sec:evaluation} presents the evaluation results. Section~\ref{sec:conclusion} concludes this paper.

\section{Background (2 pages)}
\label{sec:background}

\subsection{Graph Neural Networks}
\label{sec:gnn}

GCN~\cite{kipf2016semi}, SAGE~\cite{hamilton2017inductive}, GAT~\cite{velivckovic2017graph}, GIN~\cite{xu2018powerful}

\subsection{GNN Frameworks}
\label{sec:framework}

PyG~\cite{fey2019fast}, DGL~\cite{wang2019deep}


\section{GNN DSL (1.5 pages)}
\label{sec:gnndsl}
% 为GNN DSL取个名字,替换标题
We use GNN DSL to describe Graph Neural Networks in a simple and unified functional style.


\section{Code Transformer (1.5 pages)}
\label{sec:codetran}
% ���ڴ���������Ҫ��LLVM Clang�Դ��Ĺ��ܣ�����רע�ڴ���ת��
In this section, we propose a code transformer based on LLVM Clang to transform GUC DSL codes into executable Python codes of DGL and PyG. In order to fairly compare performance of GNN models in different GNN frameworks (e.g. DGL and PyG), the core idea of the code transformer is to obtain networks in different GNN frameworks that share the same structure as defined in GUC DSL. The solution is to transform the interface defined in GUC DSL (e.g. GCNConv) into corresponding interfaces (e.g. GraphConv in DGL and GCNConv in PyG) with necessary parameters to guarantee the same and function.

To transform codes, we use Clang, the frontend of LLVM compiler, to modify the origin code. Fig.\ref{fig:gcn_python} shows an example of how our code transformer work on the GUC DSL code in Fig.\ref{fig:gcn_guc}. We mark the important revised part, revealing that code transformer adds some fundamental syntax details and reduce the unnecessary part to change GUC DSL into executable Python code. More importantly, code transformer automatically replace critical GNN interfaces by the exact interfaces in GNN frameworks with corresponding parameters to keep same function. 

\lstset{
emph={self,GraphConv,nn,bias,norm,normalize,g,edge_index},
emphstyle=\color{blue}
}
\begin{figure}[!ht]
  \centering
  \begin{minipage}[c]{1\textwidth}
    \begin{lstlisting}
class NET(nn.Module):
  def __init__(self,in_feat,out_feat):
    super().__init__()
    self.conv1=GraphConv(in_feat,16,bias=True,norm=True)
    self.conv2=GraphConv(16,out_feat,bias=False,norm=True)
    self.relu1=nn.ReLU()
    self.relu2=nn.LeakyReLU(0.5)

  def forward(self, g, h):
    h=self.relu1(self.conv1(g,h))
    h=self.relu2(self.conv2(g,h))
    return h;
    \end{lstlisting}
    \subcaption{\centering GCN implementation in DGL}
  \end{minipage}

  \begin{minipage}[c]{1\textwidth}
    \begin{lstlisting}
class NET(nn.Module):
  def __init__(self,in_feat,out_feat):
    super().__init__()
    self.conv1=GCNConv(in_feat,16,bias=True,normalize=True)
    self.conv2=GCNConv(16,out_feat,bias=False,normalize=True)
    self.relu1=nn.ReLU()
    self.relu2=nn.LeakyReLU(0.5)

  def forward(self,edge_index,h):
    h=self.relu1(self.conv1(h,edge_index))
    h=self.relu2(self.conv2(h,edge_index))
    return h
    \end{lstlisting}
    \subcaption{\centering GCN implementation in PyG}
  \end{minipage}
  \caption{Code transformation of the GCN example}
  \label{fig:gcn_python}
\end{figure}

\section{System Design (2 pages)}
\label{sec:system}

% 突出学术上的贡献,淡化系统的工程实现;详尽说明系统支持的功能
\section{Evaluation (3 pages)}
\label{sec:evaluation}

\subsection{Experiment Setup}
\label{sec:setup}

% ƫС�����ݼ�������GPU�ŵ��£�(6-8��)��full graph training���������ݼ���1-2������mini-batch training
\subsubsection{Hardware and software configurations}
We conduct evaluation on two servers equipped with two types of GPUs, as shown in Table \ref{tab:The GPUs used for evaluation}.
The rental fee of GPUs is based on Google Cloud Server~\cite{googlecloud2022}.
Both of the server are equipped with E5-2680 v4 @ 2.4Hz, 28 cores and 256 GB main memory.

\begin{table}
    \centering
    \vspace{-1cm} %调整与上文的垂直距离
    \caption{The GPUs used for evaluation}
    \label{tab:The GPUs used for evaluation}
    \setlength{\tabcolsep}{7.mm}{
        \begin{tabular}{|c|c|c|c|}
            \hline
            GPU & Mem. & TFLOPS & Rental \\
            \hline
            V100 & 16GB HBM2 & 7.8 & 2.48\$ per hour \\
            A100 & 40GB HBM2 & 9.7 & 2.93\$ per hour \\
            \hline
        \end{tabular}
    }
    \vspace{-1cm}
\end{table}



\subsubsection{GNN configurations and graph datasets}
Four typical GNN models introduced above, GAT, GCN, GIN and GraphSAGE, are used in the experiment. The GNN models both have 2 layers of graph convolution with ReLU activation function, transformed by code transformer with GUC DSL, maintaining functional equivalence between frameworks.
As shown in Table \ref{tab:Datasets used for evaluation}, the experiments involves four scales of graph datasets: small, medium, large and oversized, to improve the universality of comparison results.

\begin{table}
    \vspace{-0.8cm} %调整与上文的垂直距离
    \centering
    \caption{Datasets used for evaluation}
    \label{tab:Datasets used for evaluation}
    \setlength{\tabcolsep}{2.mm}{
        \begin{tabular}{|c|c|c|c|c|c|}
            \hline
            Dataset & \#Nodes & \#Edges & Feature Length & \#Classes & Scale\\
            \hline
            PubMed & 19,717 & 99,203 & 500 & 3 & small \\
            PPI & 56,944 & 818,716 & 50 & 121 & small \\
            arXiv & 169,343 & 1,166,243 & 128 & 40 & middle \\
            DD & 334,925 & 1,686,092 & 89 & 32 & middle \\
            COLLAB & 235,868 & 2,358,104 & 128 & 32 & middle \\
            PPA & 576,289 & 42,463,862 & 58 & 16 & large \\
            PROTEINS & 132,534 & 79,122,504 & 8 & 2 & large \\
            reddit.dgl & 232,965 & 114,615,891 & 602 & 50 & oversized \\
            Products & 2,449,029 & 123,718,280 & 100 & 47 & oversized \\
            \hline
        \end{tabular}
    }
    \vspace{-0.7cm} %调整与下文的垂直距离
\end{table}

\subsubsection{Comparison metrics}
The key metrics for evaluation are time overhead, GPU utilization and peak memory consumption which will be measured through training multiple datasets under four GNNs.
The analysis of operators focuses on the decomposition of operators' proportion which is the metric of operator breakdown.
Based on time overhead, a case study on cost-efficiency under different GPUs will compare the pure performance and cost-efficiency.

\subsection{Framework Comparison}
\label{sec:frameworkcomp}
% ע��˵����EasyCompare�ſ�����������֮���Ĺ����ԱȺͿ��ӻ��������û���������ѡ�������ܷ���
% (on V100) Compare PyG with DGL, including execution time, GPU utilization, memory usage ... 3 * 4 figures
% Analyse the reasons of experiment results, from the perspective of framework design (e.g., graph operator implementation)
\begin{figure}
    \vspace{-0.8cm} %调整图片与上文的垂直距离
    \centering

    \begin{minipage}[c]{1\textwidth}

        \includegraphics[width=\textwidth]{images/full_graph_memory.pdf}
        \subcaption{Peak memory consumption.}
        \label{fig:full_graph_memory}


        \includegraphics[width=\textwidth]{images/full_graph_time.pdf}
        \subcaption{Exection time.}
        \label{fig:full_graph_time}

        \includegraphics[width=\textwidth]{images/full_graph_gpu.pdf}
        \subcaption{GPU utilization.}
        \label{fig:full_graph_gpu}

    \end{minipage}

    \caption{Full-graph training of 128 epoches on V100, OOM means out-of-memory.}
    \label{fig:full_graph}

    \vspace{-1cm} %调整图片与下文的垂直距离
\end{figure}

\subsubsection{Full-graph Training}
As shown in Figure \ref{fig:full_graph}, in full-graph training, we measure several training metrics on GPU of 128 epoches.
For execution time, DGL is slower by a small margin (8\%-15\%, except GAT in which DGL gets 1.2$\times$ faster) in training the small and medium-sized datasets which is due to framework overhead~\cite{wang2019deep}.
For large datasets, DGL is 1.45-2.2$\times$ faster than PyG, because DGL's GSpMM operator (shown in Figure \ref{fig:op_breakdown_dgl}) avoids generating message tensors while PyG's scatter-gather operator does no avoidance (shown in Figure \ref{fig:op_breakdown_pyg}).
It also explains why PyG runs out-of-memory in GAT on large datasets due to PPA and PROTEINS are denser and have more node features.

For peak memory consumption, it is obvious that DGL has a more powerful memory management for GNNs that PyG consumes about 2-6$\times$ more memory than DGL.
DGL manages to keep a low memory footprint due to its GSpMM operator fusing the message computation with aggregation.

For GPU utilization, PyG occupies GPU more frequently and completely than DGL.
PyG achieve 1.2-1.9x of more GPU Utilization.
A deeper look at the GPU utilization curve shows that the two curves almost coincide with each other from 0 to 100\% during training.
However, because DGL often finishes training faster, the utilization will be lower when calculating the average value.



\subsubsection{Mini-batch Training} %(on V100) Compare PyG with DGL, only execution time, 1 * 4 figures
Figure \ref{fig:minibatch_time} evaluates only the execution time of mini-batch training.
We use neighbor sampling (NS) and set mini-batch size to 10240 to train 8 epoches to obtain the measurement data.
The total execution time also depends on the cost of sample preparation, including sampling operations and data movement from CPU to GPU, compared to full graph training.
For NS, sample generation and data movement can occupy up to 85\% of the total training time while GPU Utilization hovers between 0 and 30\% which accounts for the significant performance degradation.

\begin{figure}
    \vspace{-0.5cm} %调整图片与上文的垂直距离
    \centering
    \includegraphics[width=\textwidth]{images/minibatch_time.pdf}
    \caption{Mini-batch training of 8 epoches on V100, only shows execution time, OOM means out-of-memory.}
    \label{fig:minibatch_time}
    \vspace{-1cm} %调整图片与下文的垂直距离
\end{figure}


\subsection{Operator Breakdown}
\label{sec:op_breakdown}
% (on V100) PyG, DGL, Top-10/20 operator intesection (4-6 operators each figure), 2 * 4 figures
For the DGL and PyG, we counted the proportion of operators in their respective training process and selected the top 5 operators respectively for horizontal comparison.
As shown in Figure \ref{fig:op_breakdown}, DGL has obvious characteristics in selecting operators.
For small and medium-sized data sets, the occupancy of GSpMM operators is about 20\%, and for denser datasets, it reaches 60\%-80\%.
The top 3 operators of DGL account for almost 60\% or even more than 80\%.
The operator proportion of PyG is relatively dispersed and PyG shows more scatter-gather operators while dealing with denser datasets.
Due to the gigantic real world graph, DGL developed fused message passing technique and consolidated it with sparse matrix computation into generalized sparse-dense matrix multiplication (g-SpMM) in order to reducing memory traffic.
It accounts for that DGL can achieves much more better performance (in execution time and peak memory consumption) in denser and more edge features datasets.

In addition, we found that training in GAT performs worse than other GNNs (i.e. high peak memory consumption, execution time and dispersive ratio of operators) in Figure \ref{fig:full_graph} and Figure \ref{fig:op_breakdown}.
\textsl{GATConv} scales bad by design, as autograd needs to hold tensors of shape \textsl{[num\_edges, num\_heads, 2*num\_features]} in memory.
There is no real solution to prevent this, as we need to compute attention scores for each edge and each head, based on pair-wise node features.
All conv operators will scale up by edges if they depend on pair-wise node features or edge features.
However, in \textsl{GCNConv}, it can be implemented much more memory-friendly as this operator does not require mapping node features into edge space, e.g., by using sparse-matrix multiplications.

\begin{figure}
    \vspace{-0.5cm} %调整图片与上文的垂直距离
    \centering
    \begin{minipage}[c]{1\textwidth}
        \includegraphics[width=\textwidth]{images/op_breakdown_dgl.pdf}
        \subcaption{Operator breakdown of DGL.}
        \label{fig:op_breakdown_dgl}

        \includegraphics[width=\textwidth]{images/op_breakdown_pyg.pdf}
        \subcaption{Operator breakdown of PyG.}
        \label{fig:op_breakdown_pyg}
    \end{minipage}
    \caption{Operator breakdown of training profile on V100, 128 epoches, OOM means out-of-memory.}
    \label{fig:op_breakdown}
    \vspace{-1cm} %调整图片与下文的垂直距离
\end{figure}

\subsection{Case Study: Cloud GPU Selection}
\label{sec:casestudy}
% ע��˵����EasyCompare���Է�����������ͬGPU�����ܶԱȣ������û�����GPUѡ��
% (only full-graph training) Compare V100 with A100, considering pure performance and cost efficiency, 2 * 4 figures
% Analyse the reasons of experiment results, from the perspective of hardware details, computational characteristics of GNNs
Currently, most researchers consider renting cloud GPU servers to run deep learning tasks.
This case study will simply explain the computing characteristics of GNNs and the selection of cloud servers based on the metric results of two different GPU in Table \ref{tab:The GPUs used for evaluation}.

\begin{figure}
    \vspace{-0.8cm} %调整图片与上文的垂直距离
    \centering
    \begin{minipage}[c]{1\textwidth}
        \includegraphics[width=\textwidth]{images/pure_performance.pdf}
        \subcaption{Pure performance.}
        \label{fig:pure_performance}

        \includegraphics[width=\textwidth]{images/cost_efficiency.pdf}
        \subcaption{Cost-efficiency analysis.}
        \label{fig:cost_efficiency}
    \end{minipage}
    \caption{128 epoches of training on V100 and A100 based on PROTEINS.}
    \label{fig:case_study}
    \vspace{-0.8cm} %调整图片与下文的垂直距离
\end{figure}

\subsubsection{Pure Performance}
Figure \ref{fig:pure_performance} shows the pure performance of V100 and A100 based on PROTEINS (There is no comparison of small or medium-sized datasets because they almost perform the same and except GAT because of PyG's out-of-memory).
A100 is 1.2$\times$ faster than V100 on average which confirms improvement of GPU TFLOPS.
Due to the computing characteristics of GNN are memory intensive, GPU utilization on V100 is lower than that on A100.


\subsubsection{Cost Efficiency}
Based on the pure performance and the rental price in Table \ref{tab:The GPUs used for evaluation}.
We present the cost-efficiency of the two GPUs.
Figure \ref{fig:cost_efficiency} shows the ground truth on training GNN considering cost-efficiency.
The GPU utilization is also present at the top of each bar.
It shows that the two GPUs perform different considering each GNN framework.
DGL on V100 performs better than A100 while PyG on A100 performs better than V100.

\section{Conclusion (0.3 pages)}
\label{sec:conclusion}
\textcolor{blue}{In this paper, we present an integrated system EasyCompare to conduct fair performance comparison of GNN models on different frameworks and GPUs. Easycompare contains a DSL with a code tranformer for GNN model definition, and a task running system for model training and log collection. Extensive experiment conducted with Eastcompare shows the different performance of different GNNs, frameworks and GPUs.}

\textcolor{blue}{For future work, we would extend EastCompare to support more GNNs and frameworks for better comparison. In addition, we would like to specify more details of the comparison result and add more comparison settings (e.g. comparison over multi-GPUs) to explore the factors influencing the performance of GNNs.}

%
%
%
%
% ---- Bibliography ----
%
% BibTeX users should specify bibliography style 'splncs04'.
% References will then be sorted and formatted in the correct style.
%
% \bibliographystyle{splncs04}
% \bibliography{mybibliography}
%
\bibliographystyle{splncs04}
\bibliography{mybibliography}
\end{document}
